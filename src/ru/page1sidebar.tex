\cvsection{КУРСЫ, ПРОЕКТЫ}

\cvproject{Machine Learning}{курс МФТИ, Лектор: Воронцов К.В.}{\href{https://github.com/Panterrich/MachineLearning}{Github}}
Изучил основы машинного обучения для дальнейшего применения в своих исследованиях.
\medskip

\cvproject{Разработка на GoLang}{курс МФТИ}{\href{https://github.com/Panterrich/Gitfame}{Github}}
Курс основан на \href{https://shad.yandex.ru}{ШАД Yandex}. Gitfame - основной проект этого курса.
\medskip

\cvproject{Concurrency}{курс МФТИ, Лектор: Липовский Р.Г.}{}
В курсе вводится новый взгляд на асинхронное программирование.
Курс покрывает примитивы синхронизации, планировщики, функциональные future, файберы и корутины.
\medskip

\cvproject{SSH}{}{\href{https://github.com/Panterrich/MySSH}{Github}}
Своя реализация SSH. Поддерживает симметричное и ассиметричное шифрование,
подключение нескольких пользователей, копирование файлов, а также реализацию протокола Reliable UDP.
\medskip

\cvproject{JCC}{}{\href{https://github.com/Panterrich/JCC}{Github}}
AOT компилятор, написанный для JPL. JPL - собственный тьюринг полный язык программирования.
Основной особенностью языка является двойной фронтенд. Один - для хираганы, второй - для романдзи.

\divider

\cvsection{ТЕХНИЧЕСКИЕ НАВЫКИ}
\cvskill{Языки программирования}
C, C++, Go, asm x86-64, Python
\medskip

\cvskill{Технологии}
Linux API, Boost, MPI, OpenMP
\medskip

\cvskill{Языки}
English - intermediate
\medskip

\cvskill{Другое}
Компьютерные сети, машинное обучение, вычислительная математика

\divider

\cvsection{Интересы}
\begin{itemize}
    \item Пауэрлифтинг
    \item Фотография, студенческая фотостудия
\end{itemize}

\divider

% \end{itemize}
