%%%%%%%%%%%%%%%%%
% This is an sample CV template created using altacv.cls
% (v1.1.4, 27 July 2018) written by LianTze Lim (liantze@gmail.com). Now compiles with pdfLaTeX, XeLaTeX and LuaLaTeX.
%
%% It may be distributed and/or modified under the
%% conditions of the LaTeX Project Public License, either version 1.3
%% of this license or (at your option) any later version.
%% The latest version of this license is in
%%    http://www.latex-project.org/lppl.txt
%% and version 1.3 or later is part of all distributions of LaTeX
%% version 2003/12/01 or later.
%%%%%%%%%%%%%%%%
\newcommand{\RNum}[1]{\uppercase\expandafter{\romannumeral #1\relax}}
%% If you need to pass whatever options to xcolor
\PassOptionsToPackage{dvipsnames}{xcolor}

%% If you are using \orcid or academicons
%% icons, make sure you have the academicons
%% option here, and compile with XeLaTeX
%% or LuaLaTeX.
% \documentclass[10pt,a4paper,academicons]{altacv}

%% Use the "normalphoto" option if you want a normal photo instead of cropped to a circle
\documentclass[10pt,a4paper,normalphoto]{altacv}

% \documentclass[10pt,a4paper]{altacv}

% Change the page layout if you need to
\geometry{left=1cm,right=11.2cm,marginparwidth=10cm,marginparsep=0.8cm,top=1.25cm,bottom=1.25cm,footskip=2\baselineskip}

% Change the font if you want to.

% If using pdflatex:
\usepackage[T1]{fontenc}
\usepackage[utf8]{inputenc}
\usepackage[russian]{babel}
\usepackage[default]{lato}

% Change the colours if you want to
\definecolor{Navy}{HTML}{000080}
\definecolor{SlateGrey}{HTML}{2E2E2E}
\definecolor{LightGrey}{HTML}{666666}
\definecolor{ViolentUrl}{HTML}{9D44B5}
\colorlet{heading}{Navy}
\colorlet{accent}{Navy}
\colorlet{emphasis}{SlateGrey}
\colorlet{body}{SlateGrey}

% Change the bullets for itemize and rating marker
% for \cvskill if you want to
\renewcommand{\itemmarker}{{\small\textbullet}}
\renewcommand{\ratingmarker}{\faCircle}

\usepackage[colorlinks]{hyperref}
\hypersetup{
  colorlinks,
  citecolor=ViolentUrl,
  linkcolor=Red,
  urlcolor=ViolentUrl}

\begin{document}

\name{Дурнов Алексей Николаевич}
\tagline{{\Large Исследователь \& C/C++ Разработчик (Сетевой анализ трафика)}}
%\photo{3cm}{5DSR_0451.JPG}
\personalinfo{
  \telegram{\href{https://t.me/Panterrich}{@Panterrich}}
  \github{\href{https://github.com/Panterrich}{github.com/Panterrich}}
  \linkedin{\href{htttps:://www.linkedin.com/in/aleksei-durnov-89201b216}{www.linkedin.com/in/aleksei-durnov-89201b216}}
  \newline
  \location{Moscow, Russia}
  \phone{+7(980)513-92-12}
  \email{durnov.an@phystech.edu}
  \newline
}

%% Make the header extend all the way to the right, if you want.
\begin{fullwidth}
\makecvheader
\end{fullwidth}

%% Depending on your tastes, you may want to make fonts of itemize environments slightly smaller
% \AtBeginEnvironment{itemize}{\small}


%% Provide the file name containing the sidebar contents as an optional parameter to \cvsection.
%% You can always just use \marginpar{...} if you do
%% not need to align the top of the contents to any
%% \cvsection title in the "main" bar.

\cvsection[page1sidebar]{ОБРАЗОВАНИЕ}
\cvevent{Бакалавр по профилю Радиотехника и Компьютерные технологии}{Московский физико-технический институт}{Сентябрь 2020 - Июнь 2024}{}
\begin{itemize}
    \item Средний балл: 8.8/10.0 (4.91/5.00).
    \item Топ 6 рейтинга по физтех-школе.
    \item Тема ВКР: \href{https://github.com/Panterrich/AutoSignatureGenerator}{<<Автоматическая генерация сигнатур сетевых протоколов>>}.
    Разработанные генератор сигнатур и классификатор были встроены как модули в систему анализа сетевого трафика, разрабатываемую в ИСП РАН.
\end{itemize}

\medskip

\cvevent{Магистр по профилю Радиотехника и Компьютерные технологии}{Московский физико-технический институт}{Сентябрь 2024 - Настоящее время}{}

\divider

\cvsection{ОПЫТ РАБОТЫ}

\cvevent{Лаборант (Исследователь \& C/C++ Разработчик)}{Институт системного программирования им. В.П. Иванникова Российской академии наук}{Июнь 2022 -- Настоящее время}{}
\begin{itemize}
\item Разработал модуль протокола RTSP для DPI системы \href{https://www.ispras.ru/en/technologies/protosphere/}{<<Protosphere>>}.
\item Провел \href{https://github.com/Panterrich/FTE/blob/master/fte.pdf}{обширный обзор атак протокольного туннелирования}.
\item Занимался НИР <<Анализ искаженных последовательностей>>.
Разработал верификаторы с последовательным доступом к памяти для таких форматов файлов, как zip, 7z, rar, png, tiff, rtf для системы проверки моделей.
Также реализовал множество декодер сжатых данных: Deflate, LZMA, CCITT G3, LZW, RLE, ASCII Hex, ASCII85.
\end{itemize}

\medskip

\cvevent{Ментор}{Учебный курс <<Введение в промышленное программирование на языке C>>, МФТИ, Лектор: Дединский Илья Рудольфович}{Сентябрь 2021 -- Настоящее время}{}
\begin{itemize}
    \item Курс покрывает основные структуры данных и алгоритмы, финальный проект - собственный язык программирования.
    \item Менторю группу размером 10-20 студентов каждый год.
    \item \href{https://wiki.mipt.tech/index.php/%D0%94%D1%83%D1%80%D0%BD%D0%BE%D0%B2%5F%D0%90%D0%BB%D0%B5%D0%BA%D1%81%D0%B5%D0%B9%5F%D0%9D%D0%B8%D0%BA%D0%BE%D0%BB%D0%B0%D0%B5%D0%B2%D0%B8%D1%87}{Моя страница на wiki.mipt}.
\end{itemize}

\divider

\clearpage

\end{document}
